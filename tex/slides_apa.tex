\begin{frame}[fragile]{APA - Familiar syntax/semantic to other Octave/Matlab toolboxes}

\begin{columns}
\begin{column}{0.45\linewidth}
\begin{block}{MPFR Matrix datatype and operations}
\begin{lstlisting}[basicstyle=\scriptsize,
xleftmargin=0.2cm,]
>> A = mpfr_t (eye (3)) + 2
A =
   3   2   2
   2   3   2
   2   2   3
\end{lstlisting}
\end{block}
\end{column}
\begin{column}{0.4\linewidth}
\begin{block}{Output customization}
\begin{lstlisting}[basicstyle=\tiny,
xleftmargin=0.2cm,]
>> apa ('format.base', 16)
>> a = 1 - mpfr_t (eps)
a = 0.fffffffffffff

>> apa ('format.base', 2)
>> apa ('format.fmt', 'scientific')
>> e = mpfr_t (eps)
e = 1 * 2^(-52)
\end{lstlisting}
\end{block}
\end{column}
\end{columns}

\begin{block}{Precision as binary digits (MPFR compatiblity)}
\begin{lstlisting}[basicstyle=\scriptsize,
xleftmargin=0.8cm,]
>> prec10 = 50;                     % decimal digits
>> prec2  = ceil(prec10*log2(10));  % binary  digits
>> PI = mpfr_t ('pi', prec2)
PI =
   3.1415 92653 58979 32384 62643 38327 95028 84197 16939 93751 01
\end{lstlisting}
\begin{itemize}
\item
Estimated necessary binary precision slightly larger,
50 decimal digits guaranteed correct.
\end{itemize}
\end{block}

\end{frame}


\begin{frame}[fragile]{What makes APA different?}

\begin{block}{Fast and easy installation with 5 Octave/Matlab commands.}
\begin{lstlisting}[basicstyle=\scriptsize,
xleftmargin=0.8cm,
columns=fullflexible,
upquote=true,
escapeinside={@}{@},]
urlwrite ('@{\color{magenta}https://github.com/gnu-octave/apa/releases/download/v1.0.0/apa-1.0.0.zip}@', '@{\color{magenta}apa-1.0.0.zip}@');
unzip ('@{\color{magenta}apa-1.0.0.zip}@');
cd (fullfile ('@{\color{magenta}apa-1.0.0}@', 'inst'))
install_apa
test_apa
\end{lstlisting}

\begin{itemize}
\item
Internally used GMP and MPFR libraries provided as pre-compiled static libraries.

\item
Tested and works with Octave and Matlab on \textbf{MS Windows, macOS, and Linux}.
\end{itemize}
\end{block}

\end{frame}


\begin{frame}[fragile]{What makes APA different?}
\framesubtitle{-- Work with C Code\footnote{Example
from https://www.mpfr.org/sample.html (2021-11-19).}
interactively in Octave/Matlab.}

\begin{columns}
\begin{column}[T]{0.01\textwidth}
\end{column}
\begin{column}[T]{0.3\textwidth}
\includegraphics[width=\linewidth]{res/images/mpfr_example}
\end{column}
\begin{column}[T]{0.01\textwidth}
\end{column}
\begin{column}[T]{0.38\textwidth}
\begin{lstlisting}[basicstyle=\tiny,
  columns=fullflexible,
  upquote=true,]
t = mpfr_t (1.0, 200, MPFR_RNDD);
s = mpfr_t (1.0, 200, MPFR_RNDD);
u = mpfr_t (nan, 200);

for i = 1:100
  mpfr_mul_ui (t, t, i, MPFR_RNDU);
  mpfr_set_d (u, 1.0, MPFR_RNDD);
  mpfr_div (u, u, t, MPFR_RNDD);
  mpfr_add (s, s, u, MPFR_RNDD);
end

fprintf ('Sum is ');
disp (s, MPFR_RNDD);
\end{lstlisting}

\bigskip

APA {\color{blue}Low-Level} interface:
\begin{itemize}
\item
Easy testing and debugging.

\item
Less interpreter overhead.
\end{itemize}

\end{column}
\begin{column}[T]{0.4\textwidth}

\vspace*{-1cm}
\begin{lstlisting}[basicstyle=\tiny,
  columns=fullflexible,
  upquote=true,]
setround = @(rnd) ...
mpfr_set_default_rounding_mode (rnd);
mpfr_set_default_prec (200);

setround (MPFR_RNDD)
t = mpfr_t (1.0);
s = mpfr_t (1.0);

for i = 1:100
  setround (MPFR_RNDU);
  t = t * i;
  setround (MPFR_RNDD);
  s = s + (1 / t);
end

fprintf ('Sum is ');
disp (s);
\end{lstlisting}

\bigskip

APA {\color{blue}High-Level} interface:
\begin{itemize}
\item
Mathematical language.

\item
More interpreter overhead.
\end{itemize}

\end{column}
\end{columns}

\end{frame}


\begin{frame}[fragile]{What makes APA different?}
\framesubtitle{-- Work with C Code interactively in Octave/Matlab.}

\begin{columns}
\begin{column}{0.6\linewidth}
\begin{block}{mpfr\_add.m$^{*}$}
\begin{lstlisting}[basicstyle=\tiny,escapeinside={@}{@},]
function ret = mpfr_add   (      rop,    op1,     op2,    rnd)
  ret = mex_apa_interface (@{\tt\color{orange}1031}@, @\tikzmarknode{ropm}{\tt rop.idx}@, @\tikzmarknode{op1m}{\tt op1.idx}@, @\tikzmarknode{op2m}{\tt op2.idx}@, rnd);
end
\end{lstlisting}
\end{block}

\begin{block}{mex\_apa\_interface.mex$^{*}$}{\tiny\hfill
\lstinline|mpfr_data = |
\begin{tabular}{c|c|c|c|c|c}
\hline
\ldots
& \tikzmarknode{rop}{mpfr\_t}
& mpfr\_t
& \tikzmarknode{op1}{mpfr\_t}
& \tikzmarknode{op2}{mpfr\_t}
& \ldots \\
\hline
\end{tabular}
}

\begin{lstlisting}[language=C,basicstyle=\tiny,
emph={pragma,omp,parallel,mpfr_add},emphstyle={\color{red}},
emph={[2]cmd_code,1031},emphstyle={[2]\color{orange}},
escapeinside={@}{@},]
void mexFunction ( /* ... */ ) {
  switch (cmd_code) {
    case @{\tt\color{orange}1031}@:
      #pragma omp parallel for
      for (size_t i = 0; i < length (&rop); i++)
        ret[i] = mpfr_add (rop + i, op1 + i, op2 + i, rnd);
  }
}
\end{lstlisting}
\end{block}

\begin{tikzpicture}[overlay,remember picture]
\draw[-latex,thick,blue] (ropm.south) to (rop.north);
\draw[-latex,thick,blue] (op1m.south) to (op1.north);
\draw[-latex,thick,blue] (op2m.south) to (op2.north);
\end{tikzpicture}

\vspace*{-0.5cm}
{\footnotesize{*Code simplified for presentation.}}
\end{column}
\begin{column}{0.35\linewidth}
Low-level \textbf{function calls} are fast:
\begin{itemize}
\item
Almost no m-code (less interpreter overhead).

\item
Vectorized: no loops in m-code necessary.

\item
Almost no data transfer
(two array indices for each matrix of \textbf{any size}).

\item
OpenMP \textbf{parallelization} where possible.

\item
Access to MPFR return value.
\end{itemize}
\end{column}
\end{columns}

\end{frame}


\begin{frame}[fragile]{What makes APA different?}
\framesubtitle{-- Analysis of accuracy loss in Algorithms\footnote{Experimental
feature and not contained in APA 1.0.0.}.}

\begin{columns}
\begin{column}{0.35\linewidth}
\begin{lstlisting}[basicstyle=\tiny]
>> A = mpfr_t ([2 1 1; 1 2 1; 1 1 2])
A =

   2   1   1
   1   2   1
   1   1   2

>> [L,U,~,ret] = lu (A)

L =

     1                     0   0
   0.5                     1   0
   0.5   0.33333333333333331   1

U =

   2     1                    1
   0   1.5                  0.5
   0     0   1.3333333333333333

ret =

   0   0   0
   0   0   0
   0  -1   1
\end{lstlisting}
\end{column}
\begin{column}{0.55\linewidth}
Most MPFR functions provide a ternary return value:
\begin{itemize}
\item $0$: returned value is exact.
\item $\pm 1$: returned value is greater/lower than the exact result.

\bigskip

\item
Many MPFR-based libraries ignore this value.

\item
Possibility to detect loss of accuracy in algorithms
(faster than interval arithmetic).

\item
Support design decisions about least necessary precision
for given input.
\end{itemize}
\end{column}
\end{columns}
\end{frame}